\documentclass[12pt]{article}
\usepackage{amsmath,amssymb,amsthm,mathtools}
\usepackage[T1]{fontenc}
\usepackage{lmodern}
\addtolength{\evensidemargin}{-.5in}
\addtolength{\oddsidemargin}{-.5in}
\addtolength{\textwidth}{0.8in}
\addtolength{\textheight}{0.8in}
\addtolength{\topmargin}{-.4in}
\newtheoremstyle{quest}{\topsep}{\topsep}{}{}{\bfseries}{}{ }{\thmname{#1}\thmnote{ #3}.}
\theoremstyle{quest}
\newtheorem*{definition}{Definition}
\newtheorem*{theorem}{Theorem}
\newtheorem*{question}{Question}
\newtheorem*{problem}{Problem}
\newtheorem*{answer}{Answer}
\newtheorem*{exercise}{Exercise}
\newtheorem*{challengeproblem}{Challenge Problem}
\newcommand{\name}{Ari Greenberg}
\newenvironment{sysmatrix}[1]
 {\left[\begin{array}{@{}#1@{}}}
 {\end{array}\right]}
\newcommand{\ro}[1]{%
  \xrightarrow{\mathmakebox[\rowidth]{#1}}%
}
\newlength{\rowidth}% row operation width
\AtBeginDocument{\setlength{\rowidth}{3em}}
\newcommand{\hw}{% Which homework assignment is it?
                 %
2
}
%%%%%%%%%%%%%%%%%%%%%%%%%%%%%%
\title{\vspace{-50pt}
\Huge \name
\\\vspace{20pt}
\huge 18-06 Linear Algebra\hfill Homework \hw\vspace{-30pt}}
\author{}
\date{}
\pagestyle{myheadings}
\markright{\name\hfill Homework \hw\qquad\hfill}

%% If you want to define a new command, you can do it like this:
\newcommand{\Q}{\mathbb{Q}}
\newcommand{\R}{\mathbb{R}}
\newcommand{\Z}{\mathbb{Z}}
\newcommand{\C}{\mathbb{C}}

%% If you want to use a function like ''sin'' or ''cos'', you can do it like this
%% (we probably won't have much use for this)
% \DeclareMathOperator{\sin}{sin}   %% just an example (it's already defined)


\begin{document}
\maketitle
\section*{Exercises on elimination with matrices}

\begin{problem}[2.1]
In the two-by-two system of linear equations below, what multiple of the first equation should be subtracted
from the second equation when using the method of elimination? Convert this system of equations to matrix form,
apply elimination (what are the pivots?), and use back substitution to find a solution.
Try to check your work before looking up the answer.
\begin{align*}
  2x + 3y &= 5\\
  6x + 15y &= 12
\end{align*}
\end{problem}

\begin{answer}
Multiplying the first equation by 3 and subtracting it from the second equation will eliminate x from the second equation.
Performing this operation in matrix form will result in an Upper Triangular matrix:
\begin{alignat*}{2}
\begin{sysmatrix}{rr|r}
 2 &  3 & 5 \\
 6 &  15 & 12
\end{sysmatrix}
&\!\begin{aligned}
&\ro{}\\
&\ro{r_2-3r_1}
\end{aligned}
\begin{sysmatrix}{rr|r}
 \boxed{2} &  3 & 5 \\
 0 &  \boxed{6} & -3
\end{sysmatrix}
\end{alignat*}
Back Substitution: Convert the matrix back to equations:
\begin{align*}
  6y = -3
    \implies \boxed{y = \tfrac{-1}{2}}\\
  2x + 3y = 5
    \implies 2x + 3(\tfrac{-1}{2}) = 5
    \implies 2x - \tfrac{3}{2} = 5
    \implies 2x = 5 +\tfrac{3}{2}
    \implies 2x = \tfrac{13}{2}
    \implies \boxed{x = \tfrac{13}{4}}
\end{align*}
\end{answer}

\begin{problem}[2.2]
(2.3 \#29. Introduction to Linear Algebra: Strang) Find the triangular matrix $E$ that reduces “Pascal’s matrix”
to a smaller Pascal:
\begin{alignat*}{1}
E
\begin{sysmatrix}{rrrr}
 1 & 0 & 0 & 0\\
 1 & 1 & 0 & 0\\
 1 & 2 & 1 & 0\\
 1 & 3 & 3 & 1
\end{sysmatrix}
&=
\begin{sysmatrix}{rrrr}
 1 & 0 & 0 & 0\\
 0 & 1 & 0 & 0\\
 0 & 1 & 1 & 0\\
 0 & 1 & 2 & 1
\end{sysmatrix}
\end{alignat*}
Which matrix $M$ (multiplying several $E$’s) reduces Pascal all the way to $I$?
\end{problem}
\begin{answer}
Call "Pascal's Matrix" $P$ and the smaller Pascal $Q$. We want to find $E$ such that $EP=Q$.
Note that the first row of $Q$ is identical to the first row of $P$ and each subsequent row of $Q$ is equal
to the corresponding row of $P$ minus the previous row of $P$. Thus, $E$ is a difference matrix:
\begin{alignat*}{1}
E &=
\begin{sysmatrix}{rrrr}
 1 & 0 & 0 & 0\\
 -1 & 1 & 0 & 0\\
 0 & -1 & 1 & 0\\
 0 & 0 & -1 & 1
\end{sysmatrix}
\end{alignat*}
To find $M$, we note that to convert $P$ to $I$, $I_1=P_1$, $I_2=P_2-P_1$,
$I_3=P_3-2P_2+P_1$ and $I_4= P_4-3P_3+3P_2-P_1$. Hence, we can determine the inverse matrix M:
\begin{alignat*}{1}
M &=
\begin{sysmatrix}{rrrr}
 1 & 0 & 0 & 0\\
 -1 & 1 & 0 & 0\\
 1 & -2 & 1 & 0\\
 -1 & 3 & -3 & 1
\end{sysmatrix}
\end{alignat*}
\end{answer}

\end{document}
